\documentclass[a4paper,12pt]{article}
\usepackage[T2A]{fontenc}
\usepackage[utf8]{inputenc}
\usepackage[english,russian]{babel}
\usepackage{amsmath,amsfonts,amssymb,amsthm,mathtools,mathtext}
\usepackage{graphicx}

\begin{document}
 \begin{equation}
 \sigma_R = \sqrt{(\sigma^{\text{сл}}_R)^2+(\Delta^{\text{сист}}_R)^2}
 \end{equation}  
 
 \begin{equation}
 \lim_{x\to 0} \frac{\sin(x)}{x}=1
 \end{equation}  

Правило дифференцирования сложной функции.
Производная сложной функции равна произведению производной этой функции по промежуточному аргументу и производной промежуточного аргумента по основному аргументу.
>
    Пусть $y = f(u(x)) = (f o u)(x)$. Придадим фиксированному значению аргумента $x$ приращение $\Delta{x}$. Этому приращению соответствует приращение $\Delta{u}$ функции $u(x)$. Приращению $\Delta{u}$ в свою очередь соответствует приращение $\Delta{y}$ функции $y = f(u)$ в точке $х$.
    Составим отношение
    \begin{equation}
        \frac {\Delta{y}} {\Delta{x}} = \frac {f(u) - f({u}_0)} {u - {u}_0} \cdot \frac {u - {u}_0} {x - {x}_0}
    \end{equation}  
    Аналогичная форма записи этого равества будет
    \begin{equation}
        \frac {\Delta{y}} {\Delta{x}} = \frac {\Delta{f}} {\Delta{x}}
    \end{equation}
    Так как
    \begin{equation}
        \lim_{\Delta{x}\to 0} \Delta{u} = 0,  \lim_{\Delta{x}\to 0} \Delta{y} = 0,  
        \lim_{\Delta{x}\to 0} \frac{\Delta{u}} {\Delta{x}} = u',  \lim_{\Delta{x}\to 0} \frac{\Delta{f}} {\Delta{x}} = {{f}_u}'
    \end{equation}
    так как функции $u$ и $y$ дифференцируемы. А следовательно и непрерывны.
    Получаем
    \begin{equation}
        y = f(u(x)) => y' = {{f}_u}'(u) \cdot u'(x)
    \end{equation}
    Функцию $u$ иногда называют промежуточным аргументом, а $x$ - основным аргументом.
<
 
\end{document}
\grid
